%% PARTIE 4 : Avancement du projet

%% 4.1. Product Backlog
%% 4.2. US développées 

\subsection{Product Backlog}

Les entretiens avec le client nous ont permis d'élaborer un Product Backlog qui décrit une version actuelle de ses besoins et les différentes fonctionnalités attendues pour le logiciel. Il a été changé au fur et à mesure que les besoins et exigences du client ont évolué et après réunions entre les membres de l'équipe pour rajouter de nouveaux éléments.\\

Les premières fonctionnalités que nous avons choisi d'implémenter se décomposent sous 3 parties : 
\begin{enumerate}
    \item \textbf{Création d'un plugin Python pour Inkscape}: Ce plugin se chargera d'envoyer les fichiers SVG depuis Inkscape à l'exécutable, et permettra ensuite la visualisation du résultat du packing dans Inkscape.
    \item \textbf{Gestion Entrée/Sortie SVG}: Permet de lire un fichier SVG (parser) en entrée, de l'interpréter pour pouvoir ensuite le manipuler dans le Solveur. Cette partie comprend d'autre part la génération d'un fichier SVG à partir d'une représentation interne.
    \item \textbf{Solveur : Algorithme de packing basique}: C'est une version basique d'algorithme de packing qui se charge de packer des rectangles sans grande optimisation.
\end{enumerate}


Les tableaux ci-après représentent deux versions de notre Product Backlog.

\begin{table}[h!]
\centering
\caption{Premier Product Backlog}
\label{my-label}
%\begin{adjustbox}{width=1.2\textwidth,center=\textwidth}
\resizebox{\textwidth}{!}{

\begin{tabular}{|c|c|c|}
\hline
\rowcolor[HTML]{34CDF9} 
{\color[HTML]{333333} Elément} & {\color[HTML]{333333} Option} & {\color[HTML]{333333} Description}\\ \hline
                         
                        Plugin Python (Base) & Envoi du fichier Courant & Envoyer le fichier SVG courant au parser         \\ \hline
                       
                        Pugin Python (Base)    &  Envoi de plusieurs fichiers & Envoyer plusieurs fichier SVG à la fois au parser            \\ \hline
                        
                        
                        Pugin Python (Base)     &  Visualisation & Visualiser le résultat du packing sur Inkscape            \\ \hline
                        Plugin Python (Avancé)     & Option de quantité  &  Préciser la quantité pour \\
                        & & chaque pièce sélectionnée (pour réutiliser des pièces).           \\ \hline
                         Plugin Python (Avancé)   & Option taille de plaque & Pouvoir choisir la taille de la plaque             \\ \hline
                        Gestion IO & Sortie SVG &  Transformer la représentation interne \\
                        && au niveau du solver en représentation SVG         \\ \hline
                        Gestion IO & Entrée SVG     & Interprétation du fichier SVG par un parser             \\ \hline
                        Algorithme de Packing & Algorithme basique & Algorithme de packing rectangle\\
                        && sans grande optimisation            \\ \hline
\end{tabular}}
\end{table}


\begin{table}[h!]
\centering
\caption{Product Backlog actuel}
\label{my-label}
\resizebox{\textwidth}{!}{
\begin{tabular}{|c|c|c|}
\hline
\rowcolor[HTML]{34CDF9} 
{\color[HTML]{333333} Elément} & {\color[HTML]{333333} Option} & {\color[HTML]{333333} Description}\\ \hline
                            Plugin Python(Avancé)  & Option écartement minimal & Préciser via le plugin python \\
                            && l'espace minimal entre les pièces pour la découpe.          \\ \hline
                             Algorithme de packing & Comparaison de la compression & Comparer l'efficacité des \\
                             &&algorithmes et choisir la meilleure solution.              \\ \hline
                             Algorithme de packing & Comparaison de la rapidité & Afficher des statistiques temporelles \\
                             &&sur l'exécution des algorithmes.              \\ \hline
                         
\end{tabular}}
\end{table}



%%%%%%%%%%%%%%%%%%%%%% US développées %%%%%%%%%%%%%%%%%%%%%%
\newpage
\subsection{User Stories développées}

Les éléments du Product Backlog ont été découpées en User Stories, puis mises à la norme INVEST. Le tableau suivant contient la liste des US qui ont été développées à ce jour et leur description.

Pour l'écriture de nos User Stories, nous avons identifié deux acteurs principaux : 
\begin{itemize}
    \item \textit{Robert} désigne l'utilisateur depuis le logiciel Inkscape,
    \item \textit{David} l'utilisateur depuis l'invite de commande.
\end{itemize} 

\begin{table}[H]
\centering
\caption{User Stories validées}
\label{my-label}
\resizebox{\textwidth}{!}{
\begin{tabular}{|c|c|c|}
\hline
\rowcolor[HTML]{34CDF9} 
{\color[HTML]{333333} US} & {\color[HTML]{333333} Partie} & {\color[HTML]{333333} Description}\\ \hline
                        Squelette Inkscape & Plugin Inkscape & Réaliser un squelette Inkscape \\ \hline
                        Sélection Utilisateur & Plugin Inkscape & Récupérer la sélection de l'utilisateur \\\hline
                        Plugin minimal & Plugin Inkscape & En tant que Robert je veux pouvoir envoyer\\
                        && plusieurs path fermés sélectionnés dans Inkscape afin de \\
                        &&les récupérer traités par l'exécutable dans le fichier courant.\\ \hline
                         Architecture du code &  &  Communication entre les modules  et structure\\
                        && commune \\ \hline
                        Algorithme first-fit &&  \\ \hline
                         && En tant que David je veux pouvoir packer \\
                         Gestion des groupes  &&différents objets d'un même groupe\\
                        && comme une entité unique. \\ \hline
                         && En tant que David je veux que la sortie \\
                        SVG Similaire&&du packer  possède les mêmes caractéristiques\\
                        && que le fichier d'entrée \\
                        && (en-tête et attributs des formes). \\ \hline
                         && En tant que Robert je veux que \\
                        Ajout en bas de page &&le résultat du packing s'affiche en\\
                        && bas de ma page (en dupliquant les formes) \\ \hline
                        && En tant que David, je veux pouvoir utiliser\\
                        Multi-plateforme  && le programme sur mes OS \\
                        &&préférés : Windows, Linux. \\ \hline
                        Taille plaque && En tant que Robert je veux pouvoir \\
                        && choisir la taille de plaque de découpe. \\ \hline
                        Insertion Matrice && Pouvoir insérer des matrices dans le fichier d'entrée \\ \hline
                        && Ne pas considérer les layers comme des groupes \\ \hline
                        Algorithme naïf & Algorithme &  Algorithme de packing sans optimisation \\
                        &  de packing & sans chevauchement \\ \hline
                        Détection des groupes & Parser SVG & Parser la balise de groupe pour regrouper\\
                        && les formes associées et isoler les trous. \\ \hline
                        Déplacement des groupes & Parser SVG & Gérer le déplacement simultané des groupes \\ \hline
                        Parser XML & Parser SVG & Récupération des balises path de chaque objet \\ \hline
                        Matrice de transformation & Parser SVG & Gestion des matrices de transformation \\
                        && au niveau du parser SVG \\ \hline
                        
                                               
\end{tabular}}
\end{table}


\newpage
Le tableau suivant contient les US encore non réalisées de notre liste, avec leur description.
\begin{table}[H]
\centering
\caption{User stories non validées}
\label{my-label}
\resizebox{\textwidth}{!}{
\begin{tabular}{|c|c|c|}
\hline
\rowcolor[HTML]{34CDF9} 
{\color[HTML]{333333} US} & {\color[HTML]{333333} Option} & {\color[HTML]{333333} Description}\\ \hline
                             &  & En tant que David je souhaite pouvoir rentrer\\
                             Unité dimension & Plugin Inkscape& les dimensions de plaque dans différentes unités de\\
                            && mesure  afin de pouvoir adapter le packing à mon travail. \\ \hline
                             & & En tant que David, je veux pouvoir choisir\\
                             Param. par défaut &Plugin Inkscape &les paramètres par défaut (taille du fichier) pour\\
                              &&la dimension de la plaque de manière explicite, \\
                             &&afin de ne pas renseigner de dimensions particulières. \\ \hline
                             Association objet & Algorithme de & Associer les formes les plus creuses \\
                             & packing &(aire forme/aire bounding box) deux à deux \\
                             &&pour réduire l'aire de la somme des bounding boxes. \\ \hline
                             Amélioration & Algorithme de  &   Rotation de la forme pour trouver\\
                             Bounding Box & packing &   la boite englobante la plus petite. \\ \hline
                             Brute force & Algorithme de & Tester toutes les permutations dans \\
                             Scanline & packing & l'algorithme du scanline et renvoyer le meilleur \\
                             & &  résultat de packing. \\ \hline
                              & & En tant que Robert je veux pouvoir tester \\
                             Test Tankgram & Algorithme de& l'efficacité d'un algorithme en découpant une\\
                             & Packing &forme connue en objets et en re-packant ces objets. \\ \hline
                             Séparation groupe& & Séparer les plaques en sortie en différents groupes \\ \hline
                             Distance entre les objets & & En tant que David je souhaite \\
                             && spécifier une distance de sécurité entre deux formes \\ \hline
                             Stabilité Packing & & Assurer la stabilité du packing 
                             Optimisation Scanline \\ \hline
                             Espacement plaque && Créer un espace entre l'affichage des plaques\\
                             En Sortie && \\ \hline
                             
                             
                            
\end{tabular}}
\end{table}

