%Introduction

Dans le cadre de l'enseignement de 2ème année informatique de l'ENSEIRB-MATMECA, nous avons été amenés à travailler en équipe de sept étudiants sur un projet de longue durée. L'objectif était d'une part de développer nos compétences techniques, mais également, et principalement, de nous confronter à des problématiques s'apparentant à celles rencontrées en entreprise, telles que le contact client et la méthodologie de travail de groupe.

L'utilisation de machines de découpe de matériaux telle que la découpeuse laser est de plus en plus répandue parmi les \texttt{fablabs}, leur permettant un prototypage simple, rapide et automatisé. Inévitablement, un besoin d'économiser le matériau naît, et avec lui la volonté d'optimiser la disposition des pièces à découper sur la surface donnée. Comme ce n'est pas évident de réaliser cette tâche manuellement, nous avons été chargés de trouver une solution pour l'automatiser de la meilleure manière possible.

Ce rapport présente dans un premier temps le contexte de ce projet et la méthodologie de travail que nous avons adoptée, puis dans un second temps la réponse que nous avons apportée à ce problème.

