\newpage
 
 $$(1-t)^2 =  1 -2t +t^2$$
 
 $$3b_y (1-t)^2t + 3c_y (1-t)t^2 +d_yt^3 - u_y - (3b_x(1-t)^2t + 3c_x(1-t)t^2 + d_x t^3 - u_x) \frac{v_y - u_y}{v_x - u_x}$$
 
 Developpement
 
 $$3b_y (t - 2t^2 +  t^3) + 3c_y (t^2- t^3)+ d_y t^3 - u_y - \frac{v_y - u_y}{v_x - u_x} (3b_x(t -2t^2 +t^3) + 3c_x (t^2-t^3)+ d_x t^3 - u_x)$$
 
 Facto
 $$a = (-3b_x + 3c_x - d_x)\frac{v_y - u_y}{v_x - u_x} + 3b_y - 3c_y + d_y$$
$$b = \frac{v_y - u_y}{v_x - u_x}(6b_x - 3c_x) - 6 b_y +3c_y  $$
$$ c = 3b_y - 3\frac{v_y - u_y}{v_x - u_x}b_x$$
$$ d = \frac{v_y-u_y}{v_x-ux} u_x$$


$$\Delta = b^2c^2 + 18abcd - 27a^2d^2 - 4ac^3 - 4b^3d$$

    si Δ > 0, l'équation possède trois racines réelles distinctes.\\
    si Δ = 0, l'équation possède une racine double ou triple.\\
    si Δ < 0, l'équation possède une racine réelle et deux racines complexes conjuguées.\\

$$\Delta = a^3(x_1 - x_2)^2 (x_1 - x_2)^2 (x_2 - x_3)^2$$ avec $x_1, x_2, x_3$ racines